\input{preamble.tex}

\bibliography{all}

\usepackage{subfiles}

\begin{document}
\title{Motivic Snaith Decomposition} \date{}
\maketitle

{\footnotesize
  \tableofcontents
}

\section{Introduction}

\begin{theorem}\label{thm:main}
  Over a field \(k\), there is a \(\PP^1\)--stable splitting \(\Tsusp
  \BGL_{n,+} \simeq \bigvee_{i=0}^{n-1} \Tsusp \MGL_j\).
\end{theorem}

\section{Stable Motivic Homotopy Theory of Smooth Ind-Schemes}
An ind-scheme over \(S\) is an object of \(\Ind(\Sm{S})\), the \infcat of
ind-objects in the category of smooth schemes over \(S\) with arbitrary
morphisms between them.

\(\SH\) has some six operations-like behaviour when you plug in smooth
ind-schemes \(X = \colim_i X_i\):
\[
  \SH(X) \coloneqq \lim_i \SH(X_i)
\]
along pullbacks; \(\SH(X) \simeq \colim_i \SH(X_i)\) along pushforwards in the
\infcat of presentable \infcats.\viktor{reference; this is presentable \infcat
  yoga in lurie 5.5.3, isn't it?}
\viktor{monoidal structure and dualizability is done in HA 4.6.1.11, 4.8.2.18, 3.4.3}

\viktor{formulate a proposition for the ind-scheme stuff}

For every morphism \(f\colon X\to Y\) between smooth schemes over \(S\) we have
an adjunction
\[
  \adjunction{f^*}{\SH(X)}{\SH(Y)}{f_*}
\]
between the stable presentable \infcats \(\SH(X)\) and \(\SH(Y)\). These
adjunctions assemble into functors \(\SH^*\colon \Sm{S}^\Op \to \PrL\) and
\(\SH_*\colon \Sm{S} \to \PrR\) which are naturally equivalent after composing
with the equivalence \(\PrL\simeq (\PrR)^\Op\). Because \(\PrR\) is cocomplete,
the functor \(\SH_*\) naturally extends to a functor
\[
  \SH_*\colon \Ind(\Sm{S}) \to \PrR
\]
and we obtain a functor
\[
  \SH^*\colon \Ind(\Sm{S})^\Op \to \PrL
\]
by again composing with the equivalence \(\PrL\simeq(\PrR)^\Op\).

More explicitly, if \((X_i)_{i\in I}\) is a filtered diagram of smooth schemes
over \(S\) and \(X = \colim_i X_i\) as an ind-scheme over \(S\), then
\[
  \SH^*(X) = \lim_i \SH^*(X_i)\quad\text{and}\quad\SH_*(X) = \colim_i \SH_*(X_i).
\]
Note that \(\SH^*(X)\) and \(\SH_*(X)\) are equivalent \infcats since limits
along left adjoints in \(\PrL\) correspond to colimits along their right
adjoints in \(\PrR\), see \parencite[section~5.5.3]{mr2522659}.

The adjunction \(f^*\dashv f_*\) for a morphism \(f\colon X\to Y\) of ind-schemes is obtained
by presenting \(f\) as a colimit of maps \(f_i\colon X_i\to Y_i\) between
schemes over \(S\) and then taking \(f^*\) to be the functor induced on the limits in \(\PrL\)
and \(f_*\) the functor induced on the colimits in \(\PrR\).

It remains to construct the extra left-adjoint \(f_\#\) for a smooth map \(f\)
between ind-schemes over \(S\). First, a morphism \(f\colon X\to Y\) between
ind-schemes is smooth if and only if it is a filtered colimit of smooth maps
\(f_i\colon X_i \to Y_i\)\viktor{this is a definition, isn't it?}. Each
\(f_i^*\) admits a left adjoint \(f_{i\#}\) and since \(\PrR\) is stable under
limits, the functor \(f^*\colon \SH^*(Y)\to\SH^*(X)\) admits a left adjoint as
well. That is to say, \(\SH^*\colon\Ind(\Sm{S})^\Op\to\PrL\) restricts to a
functor \(\SH^*\colon\Ind(\Sm{S})_{\sm}^\Op \to \PrR\) from the wide subcategory
of \(\Ind(\Sm{S})\) consisting of smooth maps between smooth ind-schemes over
\(S\). Composing with the equivalence \(\PrL\simeq (\PrR)^\Op\) then yields the
functor
\[
  \SH_\#\colon\Ind(\Sm{S})_\sm\to\PrL.
\]

Following \parencite{arxiv180610108L}, for a smooth morphism \(f\colon X\to Y\)
of ind-schemes over \(S\) we define \(X/Y = f_\#(\one{X})\in\SH(Y)\) where
\(\one{X}\) denotes the monoidal unit in \(\SH(X)\). In particular, if \(Y=S\),
we see that any ind-scheme \(X\) over \(S\) determines an object
\(X/S\in\SH(S)\).

\begin{proposition}\label{prop:objects-of-slash-S}
  Suppose an ind-scheme \(X\) is presented as a colimit \(X = \colim_i X_i\) in
  \(\Ind(\Sm{S})\). Then there is a natural equivalence \(X/S\simeq \colim_i
  X_i/S\) in \(\SH(S)\).
\end{proposition}
\begin{proof}
  Write \(\pi\colon X\to S\) and \(\pi_i\colon X_i\to S\) for the structure
  morphisms. Suppose \(Y\in\SH(S)\) is arbitrary. Then we have natural
  equivalences
  \begin{align*}
    \map_{\SH(S)}(\pi_\#\one{X}, Y) &\simeq \map_{\SH(X)}(\one{X}, \pi^*Y) \\
                                    &\simeq \lim_i\map_{\SH(X_i)}(\one{X_i}, \pi_i^*Y) \\
                                    &\simeq \lim_i \map_{\SH(S)}(\pi_{i\#}\one{X_i}, Y) \\
                                    &\simeq \map_{\SH(S)}(\colim_i X_i/S, Y)
  \end{align*}
  of mapping spaces. The Yoneda lemma implies that \(X/S =\pi_\#\one{X}\simeq \colim_i
  X_i/S\) in \(\SH(S)\).
\end{proof}

This proposition allows us to extend the definition of the functor
\(\_/S\colon\Sm{S}\to\SH(S)\) in \parencite{arxiv180610108L} to ind-schemes. The
functor \(\_/S\colon\Sm{S}\to\SH(S)\) extends essentially uniquely to a functor
\(\_/S\colon \Ind(\Sm{S})\to\SH(S)\) because \(\SH(S)\) is cocomplete. By
\autoref{prop:objects-of-slash-S} this coincides on objects with the previous
construction \(\pi_\#(\one{X})\) for a smooth ind-scheme \(\pi\colon X\to S\).

\section{Becker--Gottlieb Transfers in Motivic Homotopy Theory}

\begin{definition}
  In a symmetric monoidal \infcat \(\cal C\), suppose that an object \(X\) is
  equipped with a map \(\Delta\colon X\to X\otimes C\) for some other object \(C\).
  Furthermore, suppose that \(X\) is strongly dualizable. The \emph{transfer of
    \(X\) with respect to \(\Delta\)} is defined as the composition
  \[
    \transfer[X,\Delta]\colon \one{} \to[\coev] X\otimes\dual{X} \to[\switch]
    \dual{X}\otimes X\to[\id\otimes\Delta] \dual{X}\otimes X\otimes
    C\to[\eval\otimes\id] \one{}\otimes C\wequiv C.
  \]
  If there can be no risk of confusion we sometimes write \(\transfer[X] = \transfer[X,\Delta]\).
\end{definition}

\begin{theorem}[{\parencite[Theorem~1.9]{MR1867203}}]\label{thm:transfer-additivity}
  Let \(\cal C\) be a symmetric monoidal stable \infcat and let \(X\to Y\to Z\)
  be a cofiber sequence in \(\cal C\). Assume that \(C\in\cal C\) is such that
  \(\_{\otimes}C\) preserves cofiber sequences. Suppose that \(Y\) is equipped
  with a map \(\Delta_Y\colon Y\to Y\otimes C\) and that \(X\) and \(Y\) are
  strongly dualizable. Then \(Z\) is strongly dualizable and there are maps
  \(\Delta_X\) and \(\Delta_Z\) such that
  \[
    \begin{tikzcd}
      X \ar[r] \ar[d, "\Delta_X"] & Y \ar[r] \ar[d,"\Delta_Y"] & Z \ar[d,"\Delta_Z"] \\
      X\otimes C \ar[r] & Y\otimes C \ar[r] & Z\otimes C
    \end{tikzcd}
  \]
  commutes. Furthermore, we have \(\transfer[Y,\Delta_Y] = \transfer[X,\Delta_X] +
  \transfer[Z,\Delta_Z]\) in \(\pi_0\map_{\cal C}(\one, C)\).
\end{theorem}

Following \parencite{arxiv180610108L}, for a smooth map \(f\colon E\to B\)
between smooth ind-schemes over \(S\) with
\(E/B\in\SH(B)\) strongly dualizable we define the \emph{relative transfer}
\(\Tr{f/B}\colon\one{B}\to E/B\) as follows: Applying \(f_\#\) to the diagonal
\(E\to E\times_B E\) gives a morphism \(\Delta\colon E/B\to E/B\smashp E/B\) in
\(\SH(B)\) and we set \(\Tr{f/B} = \transfer[E/B,\Delta]\).

Additionally, since \(\pi\colon B\to S\) is a smooth ind-scheme, we can define
the \emph{absolute transfer} of \(f\) as
\[
  \Tr{f/S} = \pi_\#(\Tr{f/B})\colon E/S\to B/S.
\]

\begin{lemma}\label{lem:transfer-natural}
  Let \(S\) be a scheme and \(B\) and \(B'\) smooth ind-schemes over \(S\).
  Suppose that \(f\colon E\to B\) is smooth with \(E/B\in\SH(B)\) strongly
  dualizable and
  \[
    \begin{tikzcd}
      E' \ar[r, "i'"]\ar[d, "f'"] & E \ar[d, "f"] \\
      B' \ar[r, "i"'] & B
    \end{tikzcd}
  \]
  is cartesian. Then \(i^*(E/B) = E'/B'\in\SH(B')\) is strongly dualizable as
  well and we have \(\Tr{f'/B'} = i^*\Tr{f/B}\). Furthermore, the square
  \[
    \begin{tikzcd}
      E'/S \ar[r, "i'/S"] & E/S \\
      B'/S \ar[r, "i/S"'] \ar[u, "{\Tr{f'/S}}"] & B/S \ar[u, "{\Tr{f/S}}"']
    \end{tikzcd}
  \]
  is homotopy commutative.
\end{lemma}
\begin{proof}
  The first two identities \(i^*(E/B) = E'/B'\) and \(\Tr{f'/B'} = i^*\Tr{f/B}\)
  are proven in \parencite[Lemma~1.6]{arxiv180610108L}. Write \(p\colon B'\to
  S\) and \(q\colon B\to S\) for the structure morphisms. There is a natural
  transformation \(p_\#i^*\to q_\#\) defined as the composition
  \[
    p_\#i^* \to[\unit] p_\#i^*q^*q_\# \simeq p_\# p^*q_\# \to[\counit] q_\#.
  \]
  Consequently, we obtain a homotopy commutative diagram
  \[
    \begin{tikzcd}[column sep=5em]
      \mathllap{B'/S = {}}p_\#p^*\one{S} \ar[r, "p_\#\Tr{f'/B'}"] \ar[d, equal] & p_\#f'_\#
      {i'}^*f^*q^*\one{S}\mathrlap{{} = E'/S}\ar[d, "\Ex_\#^*"] \\
      p_\#i^*q^*\one{S} \ar[r, "p_\#i^*\Tr{f/B}"] \ar[d] &
      p_\#i^*f_\#f^*q^*\one{S} \ar[d] \\
      \mathllap{B/S = {}}q_\# q^*\one{S} \ar[r, "q_\#\Tr{f/B}"'] & q_\#
      f_\#f^*q^*\one{S}\mathrlap{{} = E/S.}
    \end{tikzcd}
  \]
  Chasing through the definition of \(p_\#i^*\to q_\#\) shows that the leftmost
  composite vertical map is \(i/S\) and that the rightmost vertical map is \(i'/S\).
\end{proof}

\begin{theorem}[{\parencite[Proposition~1.2]{arxiv180610108L}}]\label{thm:local-dualizability}
  Let \(B\) be a scheme over \(S\). Suppose that \(E\in\SH(B)\) and there is a
  finite Nisnevich covering family \(\{j_i\colon U_i\to B\}\) and that \(j_i^*E
  \in\SH(U_i)\) is strongly dualizable. Then \(E\) is strongly dualizable as well.
\end{theorem}

It follows that a smooth map \(E\to B\) of schemes over \(S\) which is locally
trivial in the Nisnevich topology with smooth fiber \(F\) defines a strongly
dualizable object \(E/B\in\SH(B)\) if \(F/S\in\SH(S)\) is strongly dualizable.

\begin{lemma}\label{lem:ind-dualizability}
  Suppose \(B\) is a smooth ind-scheme over \(S\) and \(E\in\SH(B)\). If \(B\)
  is presented as a filtered colimit \(B = \colim_i B_i\) of smooth schemes in
  \(\Ind(\Sm{S})\), let \(f_i\colon B_i\to B\) be the canonical map for each
  \(i\). Then \(E\in\SH(B)\) is strongly dualizable if and only if
  \(f_i^*E\in\SH(B_i)\) is strongly dualizable for every \(i\).
\end{lemma}
\begin{proof}
  This follows from \parencite[Proposition~4.6.1.11]{higheralgebra} since we
  have \(\SH(B)\simeq \lim_i\SH(B_i)\).
\end{proof}

\section{Transfers of Grassmannians}

From now on, we assume that the base scheme \(S\) is the spectrum of a field
\(k\).

\begin{definition}
  The ind-scheme \(\Gr{r}\) is the sequential colimit of the Grassmannians
  \(\Gr[n]{r}\) of \(r\)--planes in \(n\)--space along the canonical closed immersions
  \(\Gr[n]{r}\clim\Gr[n+1]{r}\).
\end{definition}

It is well that known \(\Gr{r}\) is a model of \(\BGL_r\) in the
\(\AA^1\)--homotopy category. In fact, let \(U_r(N)\) be the scheme of
monomorphisms \(\cal O^r \to \cal O^N\). Along \(\cal O^N\oplus 0 \subset \cal
O^{N+1}\), there are closed embeddings \(U_r(N)\clim U_r(N+1)\) and
\parencite[Proposition~4.3.7]{mv} shows that the colimit \(U_r(\infty) =
\colim_N U_r(N)\) along these embeddings is contractible in \(\H(k)\). Also, the
quotient \(U_r(N)/\GL_r\) is isomorphic to \(\Gr[N]{r}\) and consequently
\(U_r(\infty)/\GL_r\isom \Gr{r}\) is a model for \(\BGL_r\).

Direct sum defines a morphism \(U_r(N)\times U_{n-r}(N)\to U_n(2N)\) which is
equivariant with respect to the block diagonal inclusion
\(\GL_r\times\GL_{n-r}\to \GL_n\). Passing to the colimit \(N\cto\infty\) and
taking quotients yields a morphism
\[
  i_{r,n}\colon\Gr{r}\times\Gr{n-r} \to \Gr{n}.
\]
The goal of this section is to develop a partial inductive description of the
absolute transfer \(\transfer[n,r]\colon
\BGL_{n,+}\to\BGL_{r,+}\smashp\BGL_{n-r,+}\) of \(i_{r,n}\) in \(\SH(k)\). For
this purpose a slightly different version of \(i_{r,n}\) in \(\H(k)\) will
be more convenient.

\begin{lemma}
  In \(\H(k)\) there is an equivalence \(\Gr{r}\times\Gr{n-r}\to
  U_n(\infty)/(\GL_r\times\GL_{n-r})\). Along this equivalence, \(i_{r,n}\)
  corresponds to the quotient
  \[
    \overline{i_{r,n}}\colon U_n(\infty)/(\GL_r\times\GL_{n-r}) \to U_n(\infty)/\GL_n\isom \Gr{n}
  \]
  by \(\GL_n\).
\end{lemma}
\begin{proof}
  Writing \(\varphi\colon U_r(N)\times U_{n-r}(N)\to U_n(2N)\) for the map
  induced by taking direct sums, we obtain a commutative diagram
  \[
    \begin{tikzcd}
      U_r(N)\times U_{n-r}(N) \ar[r, "\varphi"]\ar[d] & U_n(2N)\ar[d] \\
      \Gr[N]{r}\times\Gr[N]{n-r} \ar[r]\ar[dr, "i_{r,n}"'] & U_n(2N)/(\GL_r\times\GL_{n-r})\ar[d] \\
      & \Gr[2N]{n}.
    \end{tikzcd}
  \]
  Passing to the colimit \(N\cto\infty\) the horizontal maps become equivalences.
\end{proof}

\begin{lemma}\label{lem:gr-fiber-bundle}
  The morphism \(\overline{i_{r,n}}\) is a Zariski--locally trivial bundle over
  \(\Gr{n}\). Its fiber is the quotient \(\GL_n/(\GL_r\times\GL_{n-r})\).
\end{lemma}
\begin{proof}
  By construction, the morphism \(\overline{i_{r,n}}\) is isomorphic to the
  colimit of the quotient maps \(U_n(N)/(\GL_r\times\GL_{n-r}) \to
  U_n(N)/\GL_n\isom \Gr[N]{n}.\) But these are Zariski--locally trivial with
  fiber \(\GL_n/(\GL_r\times\GL_{n-r})\) since \(\GL_r\times\GL_{n-r}\) is an
  admissible subgroup of \(\GL_n\).
\end{proof}

We note that \(\GL_n/(\GL_r\times\GL_{n-r})\) is equivalent to \(\Gr[n]{r}\) in
\(\H(k)\) and this equivalence is compatible with the respective \(\GL_n\)
actions. This is shown in \parencite[Lemma~3.1.5]{2015arXiv150708020A} and
implies in particular that the associated bundle
\(U_n(\infty)\times^{\GL_n}\Gr[n]{r}\to\Gr{n}\) is equivalent to
\(\Gr{r}\times\Gr{n-r}\to\Gr{n}\) in \(\H(\Gr{n})\).

\begin{lemma}
  The morphism \(\overline{i_{r,n}}\colon U_n(\infty)/(\GL_r\times\GL_{n-r})
  \to\Gr{n}\) defines a strongly dualizable object in \(\Gr{r}\times\Gr{n-r}/\Gr{n}\in\SH(\Gr{n})\).
\end{lemma}
\begin{proof}
  By \autoref{lem:ind-dualizability} it will be enough to show that the
  pullback \(i\colon E\to \Gr[N]{n}\) of \(\overline{i_{r,n}}\) along the inclusion
  \(\Gr[N]{n}\to\Gr{n}\) defines a dualizable object in \(\SH(\Gr[N]{n})\) for
  all \(N\). But, by \autoref{lem:gr-fiber-bundle} the morphism \(i\) is a
  Zariski-locally trivial fiber bundle over \(\Gr[N]{n}\) with fiber
  \(X = \GL_n/(\GL_r\times\GL_{n-r})\). Hence, to show that \(i\) defines a strongly
  dualizable object in \(\SH(\Gr[N]{n})\), by \autoref{thm:local-dualizability}
  it is enough to show that \(X/k\in\SH(k)\) is strongly dualizable.

  But we have seen that \(X\simeq\Gr[n]{r}\) in \(\H(k)\) and therefore also in
  \(\SH(k)\). The scheme \(\Gr[n]{r}\) is smooth and proper, so motivic Atiyah
  duality implies that \(\Gr[n]{r}/k\) and therefore also \(X/k\) is strongly
  dualizable in \(\SH(k)\), see for example \parencite[Proposition~1.2]{arxiv180610108L}.
\end{proof}

\begin{lemma}\label{lem:grassmann-decomp}
  Suppose \(r<n\). The open complement of the closed immersion
  \(\Gr[n-1]{r}\clim \Gr[n]{r}\) is the total space of an affine space bundle of
  rank \(n-r\) over \(\Gr[n-1]{r-1}\).

  Dually, the complement of the closed immersion \(\Gr[n-1]{r-1}\clim
  \Gr[n]{r}\) is the total space of an affine space bundle of rank \(r\) over
  \(\Gr[n-1]{r}\).
\end{lemma}
\begin{proof}
  Let \(A\) be a \(k\)--algebra. On \(\Spec(A)\)--valued points, the inclusion
  \(\Gr[n-1]{r}\clim\Gr[n]{r}\) is given by considering a projective submodule
  \(P\) of \(A^{n-1}\) as a submodule of \(A^n = A^{n-1}\oplus A\). It follows
  that the complement \(U\) of \(\Gr[n-1]{r}\) has \(\Spec(A)\)--valued points
  \[
    U(\Spec A) = \{P\subset A^n : \text{\(P\) is projective of rank \(r\) and
      \(P \not\subset A^{n-1}\oplus 0\)}\}.
  \]
  Given \(P\in U(\Spec A)\), the module \(P\cap (A^{n-1}\oplus 0)\) will be
  locally free of rank \(r-1\). This gives a map \(\varphi\colon U \to
  \Gr[n-1]{r-1}\) which is trivial over the standard Zariski--open cover of
  \(\Gr[n-1]{r-1}\) with fiber \(\AA^{n-r}\).

  The dual statement is proved similarly. In fact, the bundle \(V \to
  \Gr[n-1]{r}\) in question is the tautological \(r\)-plane bundle on
  \(\Gr[n-1]{r}\).
\end{proof}

The decomposition \(\Gr[n]{r} = U\cup V\) of the last lemma yields a
homotopy cocartesian square
\[
  \begin{tikzcd}
    \mathllap{U\setminus \Gr[n-1]{r-1} ={}} U\cap V \ar[r] \ar[d] &
    V\mathrlap{{}\simeq \Gr[n-1]{r}} \ar[d] \\
    \mathllap{\Gr[n-1]{r-1}\simeq {}}U \ar[r] & \Gr[n]{r}
  \end{tikzcd}
\]
in the \(\AA^1\)--homotopy category \(\H(k)\). It is immediate that this
decomposition of \(\Gr[n]{r}\) is stable under the action of \(\GL_{n-1}\times
1\subset \GL_n\). We can therefore pass to the bundles over \(\Gr{n-1}\)
associated to the universal \(\GL_{n-1}\)--torsor \(U_{n-1}(\infty)\) over
\(\Gr{n-1}\) and obtain a homotopy cocartesian square
\[
  \begin{tikzcd}
    U_{n-1}(\infty)\times^{\GL_{n-1}} (U\cap V)\ar[r]\ar[d] & \Gr{r}\times\Gr{n-1-r} \ar[d] \\
    \Gr{r-1}\times\Gr{n-r}\ar[r] & U_{n-1}(\infty)\times^{\GL_{n-1}}\Gr[n]{r}
  \end{tikzcd}
\]
in \(\H(\Gr{n-1})\).

\begin{proposition}\label{prop:transfer-decomp}
  Suppose \(r<n\) and consider the composition
  \[
    \varphi\colon \Gr{n-1,+} \to[\incl] \Gr{n+} \to[{\transfer[n,r]}]\Gr{r+}\smashp\Gr{n-r,+}
  \]
  where \(\incl\) is given by the assignment \(P\mapsto P\oplus A\) on \(\Spec(A)\)--valued
  points. Then there is a map \(\psi\colon \Gr{n-1,+}\to
  \Gr{r-1,+}\smashp\Gr{n-r,+}\) in \(\SH(k)\) such that \(\varphi\) is the sum
  of the compositions
  \begin{align*}
    \Gr{n-1,+} \to[{\transfer[n-1,r]}]\Gr{r,+}\smashp\Gr{n-1-r,+} \to[\id\smashp\incl]\Gr{r,+}\smashp\Gr{n-r,+} \\
    \Gr{n-1,+} \to[{\transfer[n-1,r-1]}]\Gr{r-1,+}\smashp\Gr{n-r,+}\to[\incl\smashp\id]\Gr{r,+}\smashp\Gr{n-r,+} \\
\shortintertext{and}
    \Gr{r-1,+}\to[\psi]\Gr{r-1,+}\smashp\Gr{n-r,+} \to[\incl\smashp\id]\Gr{r,+}\smashp\Gr{n-r,+}.
  \end{align*}
\end{proposition}
\begin{proof}
  Consider the homotopy pullback
  \[
    \begin{tikzcd}
      \mathllap{E={}}U_{n-1}(\infty)\times^{\GL_{n-1}}\Gr[n]{r}\ar[d]\ar[r] & \Gr{r}\times\Gr{n-r}\ar[d] \\
      \Gr{n-1}\ar[r, "\incl"'] & \Gr{n}.
    \end{tikzcd}
  \]
  By the discussion following \autoref{lem:grassmann-decomp} we obtain a cofiber
  sequence
  \[
    X/\Gr{n-1} \to (\Gr{r}\times\Gr{n-1-r})/\Gr{n-1}\vee(\Gr{r-1}\times\Gr{n-r})/\Gr{n-1} \to
    E/\Gr{n-1}
  \]
  in \(\SH(\Gr{n-1})\) where \(X = U_{n-1}(\infty)\times^{\GL_{n-1}}(U\cap V)\).
  \Autoref{thm:transfer-additivity} then shows that
  \[
    \transfer[E/\Gr{n-1}] = \transfer[\Gr{r}\times\Gr{n-1-r}/\Gr{n-1}] +
    \transfer[\Gr{r-1}\times\Gr{n-r}/\Gr{n-1}] - \transfer[X/\Gr{n-1}]
  \]
  in \(\SH(\Gr{n-1})\). Passing to the absolute transfer and using
  \autoref{lem:transfer-natural} yields that \(\varphi\) is the sum of
  the compositions
  \begin{align*}
    &\Gr{n-1,+} \to[{\transfer[n-1,r]}] \Gr{r,+}\smashp\Gr{n-1-r,+} \to[\id\smashp\incl]\Gr{r,+}\smashp\Gr{n-r,+} \\
    &\Gr{n-1,+} \to[{\transfer[n-1,r-1]}] \Gr{r-1,+}\smashp\Gr{n-r,+} \to[\incl\smashp\id]\Gr{r,+}\smashp\Gr{n-r,+} \\
\shortintertext{and}
    &\Gr{n-1,+}\to X_+ \to \Gr{r,+}\smashp\Gr{n-r,+}
  \end{align*}
  in \(\SH(k)\). Here, the map \(X_+\to\Gr{r,+}\smashp\Gr{n-r,+}\) is obtained
  from the inclusion \(U\cap V\subset \Gr[n]{r}\) by passing to associated
  bundles. Now, this inclusion factors through the inclusion of \(U\) into
  \(\Gr[n]{r}\). By \autoref{lem:grassmann-decomp} the inclusion
  \(\Gr[n-1]{r-1}\subset U\) is an \(\AA^1\)--equivalence, being the zero
  section of an affine space bundle. Therefore
  \(X_+\to\Gr{r,+}\smashp\Gr{n-r,+}\) factors through the map
  \(\incl\smashp\id\colon\Gr{r-1,+}\smashp\Gr{n-r,+}\to\Gr{r,+}\smashp\Gr{n-r,+}\).
  This way we obtain the map \(\psi\) and the required decomposition of \(\transfer[n,r]\circ\incl\).
\end{proof}

\section{Proof of the Theorem}

The inclusions \(\GL_i\times 1\subset\GL_n\) induce a filtration
\[
    \BGL_{0,+}\to[i_1] \BGL_{1,+} \to[i_2] \dots \to[i_n] \BGL_{n,+} \to
    \dots\to[i_m] \BGL_{m,+}.
\]
We have seen that for \(r \leq n\) the block--diagonal inclusion
\(i_{r,n}\colon\BGL_r\times\BGL_{n-r}\to \BGL_n\) admits an absolute transfer
\(\transfer[n,r]\colon\BGL_{n,+}\to \BGL_{r,+}\smashp \BGL_{n-r,+}\) in the
motivic stable homotopy category \(\SH(k)\). Write \(f_{n,r}\colon \BGL_{n,+}\to
\BGL_{r,+}\) for the composition
\[
  \BGL_{n,+} \to[{\transfer[n,r]}] \BGL_{r,+}\smashp\BGL_{n-r,+} \to[\proj] \BGL_{r,+}
\]
and \(\phi_{n,r}\) for the composition
\[
  \BGL_{n,+}\to[f_{n,r}] \BGL_{r,+}\to\BGL_r/{\BGL_{r-1}}.
\]

\begin{lemma}\label{lem:phi-comp}
  With notation as above, for \(r < n\) the compositions
\[
  \BGL_{n-1,+} \to[i_n] \BGL_{n,+} \to[f_{n,r}] \BGL_{r,+} \to \BGL_r/{\BGL_{r-1}}
\]
and
\[
  \BGL_{n-1,+} \to[f_{n-1,r}] \BGL_{r,+} \to \BGL_r/{\BGL_{r-1}}
\]
coincide.
\end{lemma}
\begin{proof}
  By \autoref{prop:transfer-decomp} the composition \(f_{n,r}\circ i_n\) is a
  sum of two compositions
  \begin{align*}
    &\BGL_{n-1,+} \to[{\transfer[n-1,r]}] \BGL_{r,+}\smashp\BGL_{n-1-r,+} \to[\id\smashp\incl]\BGL_{r,+}\smashp\BGL_{n-r,+}\to[\proj]\BGL_{r,+} \\
    \shortintertext{and}
    &\BGL_{n-1,+}\to\BGL_{r-1,+}\smashp\BGL_{n-r,+}\to[\incl\smashp\id]\BGL_{r,+}\smashp\BGL_{n-r,+}\to[\proj]\BGL_{r,+}
  \end{align*}
  in \(\SH(k)\). But the composition
  \[
    \BGL_{r-1,+}\to[\incl]\BGL_{r,+} \to \BGL_r/{\BGL_{r-1}}
  \]
  vanishes. Therefore, \(f_{n,r}\circ i_n\) coincides with the composition
  \[
    \BGL_{n-1,+}\to[f_{n-1,r}]\BGL_{r,+}\to\BGL_r/{\BGL_{r-1}}
  \]
  in \(\SH(k)\).
\end{proof}

\begin{proof}[{Proof of \autoref{thm:main}}]
  Proceeding by induction on \(n\), assume that
\[
  \Phi = \bigvee_{r=0}^{n-1} \phi_{n-1,r}\colon \BGL_{n-1,+} \to \bigvee_{r=0}^{n-1} \BGL_r/{\BGL_{r-1}}
\]
is a \(\PP^1\)--stable equivalence. Because of \autoref{lem:phi-comp} we have a
commutative diagram
\[
  \begin{tikzcd}
    {} & \BGL_{n,+}\ar[dr, "\Phi'"] & {} \\
    \BGL_{n-1,+} \ar[rr, "\Phi"'] \ar[ur, "i_n"] & {} & \displaystyle\bigvee_{r=0}^{n-1}\BGL_r/{\BGL_{r-1}}
  \end{tikzcd}
\]
where \(\Phi' = \bigvee_{r=0}^{n-1}\phi_{n,r}\). It follows that \(\Phi^{-1}\circ\Phi'\circ i_n \simeq \id\),
\ie~\(i_n\) admits a left inverse. This implies that we have a \(\PP^1\)--stable equivalence
\[
  \BGL_{n,+} \to[{(\Phi^{-1}\Phi') \vee \phi_{n,n}}] \BGL_{n-1,+} \vee \BGL_n/{\BGL_{n-1}}
\]
since \(\phi_{n,n}\) is by definition the canonical projection. Post-composing
with \(\Phi\vee{\id}\) then shows that the stable map \(\Phi'\vee\phi_{n,n}\colon
\BGL_{n,+}\to\bigvee_{r=0}^n \BGL_r/{\BGL_{r-1}}\) is a \(\PP^1\)--stable
equivalence as well.
\end{proof}

\printbibliography

\listoftodos
\end{document}
% Local Variables:
% TeX-master: t
% End:
