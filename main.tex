\input{preamble.tex}

\bibliography{all}

\usepackage{subfiles}

\begin{document}
\title{Motivic Snaith Decomposition} \date{}
\maketitle

{\footnotesize
  \tableofcontents
}

\section{Introduction}

\begin{theorem}\label{thm:main}
  Over a field \(k\), there is a \(\PP^1\)--stable splitting \(\Tsusp
  \BGL_{n,+} \simeq \bigvee_{i=0}^{n-1} \Tsusp \MGL_j\).
\end{theorem}

\section{Stable Motivic Homotopy Theory of Smooth Ind-Schemes}
\(\SH\) has some six operations-like behaviour when you plug in smooth
ind-schemes \(X = \colim_i X_i\):
\[
  \SH(X) \coloneqq \lim_i \SH(X_i)
\]
along pullbacks; \(\SH(X) \simeq \colim_i \SH(X_i)\) along pushforwards in the
\infcat of presentable \infcats.\viktor{reference; this is presentable \infcat
  yoga in lurie 5.5.3, isn't it?}
\viktor{monoidal structure and dualizability is done in HA 4.6.1.11, 4.8.2.18, 3.4.3}

For every morphism \(f\colon X\to Y\) between smooth schemes over \(S\) we have
an adjunction
\[
  \adjunction{f^*}{\SH(X)}{\SH(Y)}{f_*}
\]
between the stable presentable \infcats \(\SH(X)\) and \(\SH(Y)\). These
adjunctions assemble into functors
\[
  \SH^*\colon \Sm{S}^\Op \to \PrL
\]
and
\[
  \SH_*\colon \Sm{S} \to \PrR
\]
which are naturally equivalent after composing with the equivalence
\(\PrL\simeq (\PrR)^\Op\). Because \(\PrR\) is cocomplete, the functor \(\SH_*\)
naturally extends to a functor
\[
  \SH_*\colon \Ind(\Sm{S}) \to \PrR
\]
and we obtain a functor
\[
  \SH^*\colon \Ind(\Sm{S})^\Op \to \PrL
\]
by again composing with the equivalence \(\PrL\simeq(\PrR)^\Op\).

More explicitly, if \((X_i)_{i\in I}\) is a filtered diagram of smooth schemes
over \(S\) and \(X = \colim_i X_i\) as an ind-scheme over \(S\), then
\[
  \SH^*(X) = \lim_i \SH^*(X_i)\quad\text{and}\quad\SH_*(X) = \colim_i \SH_*(X_i).
\]
The adjunction \(f^*\dashv f_*\) for a morphism \(f\colon X\to Y\) of ind-schemes is obtained
by presenting \(f\) as a colimit of maps \(f_i\colon X_i\to Y_i\) between
schemes over \(S\) and then taking \(f^*\) to be the functor induced on the limits in \(\PrL\)
and \(f_*\) the functor induced on the colimits in \(\PrR\).

It remains to construct the extra left-adjoint \(f_\#\) for a smooth map \(f\)
between ind-schemes over \(S\). First, a morphism \(f\colon X\to Y\) between
ind-schemes is smooth if and only if it is a filtered colimit of smooth maps
\(f_i\colon X_i \to Y_i\).\viktor{this is a definition, isn't it?}

Note that \(B/S\in\SH(S)\) is by definition \(\pi_\#(\one{B})\) for \(\pi\colon
B\to S\) a smooth ind-scheme over \(S\). Also, if \(f\colon E\to B\) is a
smooth morphism, then \(E \isom \colim_k i_k^*E\) is a smooth
ind-scheme.\viktor{This should follow from weak descent in the big Zariski topos
  of \(B\); maybe?}

\begin{definition}\label{defn:extended-sharp}
  Suppose that \(B = \colim{} (B_k)_{k\in I}\) is a smooth ind-scheme over a scheme
  \(S\) and let \(\pi_k\colon B_k\to S\) and \(\pi\colon B\to S\) be the
  structure morphisms. Define
  \[
    \pi_\#(E) = \colim_{k\in I} \pi_{k\#}i_k^*(E)
  \]
  for \(E\in\SH(B)\) where \(i_k\colon B_k\to B\) is the natural map.
  This way we obtain a functor \(\pi_\#\colon\SH(B)\to\SH(S)\).
\end{definition}

\section{Becker--Gottlieb Transfers in Motivic Homotopy Theory}

We will be using the six functor formalism for the stable motivic homotopy
category established by Ayoub \viktor{reference}. Hoyois \viktor{reference} has
extended it to arbitrary schemes without noetherian assumptions. In summary,
for every scheme \(X\) there is a stable closed symmetric monoidal \infcat
\(\SH(X)\) of motivic spectra over \(X\). For every morphism of schemes
\(f\colon Y\to X\) there is an adjunction
\[
  \adjunction{f^*}{\SH(X)}{\SH(Y)}{f_*}
\]
and for \(f\) separated of finite type and \(X\) and \(Y\) quasicompact and
quasiseparated there is an additional adjunction
\[
  \adjunction{f_!}{\SH(Y)}{\SH(X)}{f^!}
\]
satisfying various axioms \viktor{add them}.

\begin{definition}
  In a symmetric monoidal \infcat \(\cal C\), suppose that an object \(X\) is
  equipped with a map \(\Delta\colon X\to X\otimes C\) for some other object \(C\).
  Furthermore, suppose that \(X\) is strongly dualizable. The \emph{transfer of
    \(X\) with respect to \(\Delta\)} is defined as the composition
  \[
    \transfer[X,\Delta]\colon \one{} \to[\coev] X\otimes\dual{X} \to[\switch]
    \dual{X}\otimes X\to[\id\otimes\Delta] \dual{X}\otimes X\otimes
    C\to[\eval\otimes\id] \one{}\otimes C\wequiv C.
  \]
  If there can be no risk of confusion we sometimes write \(\transfer[X] = \transfer[X,\Delta]\).
\end{definition}

\begin{theorem}[{\parencite[Theorem~1.9]{MR1867203}}]
  Let \(\cal C\) be a symmetric monoidal stable \infcat and let \(X\to Y\to Z\)
  be a cofiber sequence in \(\cal C\). Assume that \(C\in\cal C\) is such that
  \(\_{\otimes}C\) preserves cofiber sequences. Suppose that \(Y\) is equipped
  with a map \(\Delta_Y\colon Y\to Y\otimes C\) and that \(X\) and \(Y\) are
  strongly dualizable. Then \(Z\) is strongly dualizable and there are maps
  \(\Delta_X\) and \(\Delta_Z\) such that
  \[
    \begin{tikzcd}
      X \ar[r] \ar[d, "\Delta_X"] & Y \ar[r] \ar[d,"\Delta_Y"] & Z \ar[d,"\Delta_Z"] \\
      X\otimes C \ar[r] & Y\otimes C \ar[r] & Z\otimes C
    \end{tikzcd}
  \]
  commutes. Furthermore, we have \(\transfer[Y,\Delta_Y] = \transfer[X,\Delta_X] +
  \transfer[Z,\Delta_Z]\) in \(\pi_0\map_{\cal C}(\one, C)\).
\end{theorem}

Following \parencite{arxiv180610108L}, for a smooth map \(f\colon E\to B\) with
\(E/B\in\SH(B)\) strongly dualizable we define the \emph{relative transfer}
\(\Tr{f/B}\colon\one{B}\to E/B\) as follows: Applying \(f_\#\) to the diagonal
\(E\to E\times_B E\) gives a morphism \(\Delta\colon E/B\to E/B\smashp E/B\) in
\(\SH(B)\) and we set \(\Tr{f/B} = \transfer[E/B,\Delta]\).

If additionally \(\pi\colon B\to S\) is smooth, we can define the \emph{absolute
  transfer} of \(f\) as
\[
  \Tr{f/S} = \pi_\#(\Tr{f/B})\colon E/S\to B/S.
\]

\begin{lemma}\label{lem:transfer-natural}
  Let \(S\) be a scheme and \(B\) and \(B'\) smooth ind-schemes over \(S\).
  Suppose that \(f\colon E\to B\) is smooth with \(E/B\in\SH(B)\) strongly
  dualizable and
  \[
    \begin{tikzcd}
      E' \ar[r]\ar[d, "f'"] & E \ar[d, "f"] \\
      B' \ar[r, "i"'] & B
    \end{tikzcd}
  \]
  is cartesian. Then \(i^*(E/B) = E'/B'\in\SH(B')\) is strongly dualizable as
  well and we have \(\Tr{f'/B'} = i^*\Tr{f/B}\). Furthermore, the square
  \[
    \begin{tikzcd}
      E'/S \ar[r] & E/S \\
      B'/S \ar[r, "i"'] \ar[u, "{\Tr{f'/S}}"] & B/S \ar[u, "{\Tr{f/S}}"']
    \end{tikzcd}
  \]
  is commutative.
\end{lemma}

\begin{theorem}[{\parencite[Proposition~1.2]{arxiv180610108L}}]
  Let \(B\) be a scheme over \(S\). Suppose that \(E\in\SH(B)\) and there is a
  finite Nisnevich covering family \(\{j_i\colon U_i\to B\}\) and that \(j_i^*E
  \in\SH(U_i)\) is strongly dualizable. Then \(E\) is strongly dualizable as well.
\end{theorem}

It follows that a smooth map \(E\to B\) over \(S\) which is locally trivial in the
Nisnevich topology with smooth fiber \(F\) defines a strongly dualizable object
\(E/B\in\SH(B)\) if \(F/S\in\SH(S)\) is strongly dualizable.

\begin{lemma}\viktor{this is too cumbersome}
  Suppose \(X = U\cup V\) for \(U,V\) Zariski open in \(X\). Let
  \(\transfer[U]\colon \one{}\to U_+\) and \(\transfer[V]\colon \one{}\to V_+\)
  be the Becker--Gottlieb transfers for \(U\) and \(V\) respectively. Then there
  is a Mayer--Vietoris type decomposition for the transfer
  \(\transfer[X]\colon\one{}\to X_+\), namely
  \[
    \transfer[X] = i_{U,+}\circ\transfer[U] + i_{V,+}\circ\transfer[V] -
    i_{U\cap V,+}\circ\transfer[U\cap V]
  \]
  where \(i_U\), \(i_V\) and \(i_{U\cap V}\) are the evident inclusions and
  \(\transfer[U\cap V]\colon\one{}\to (U\cap V)_+\) is the Becker--Gottlieb
  transfer for \(U\cap V\).
\end{lemma}


\section{Equivariant Transfers of Grassmannians\viktor{change that title}}

\begin{lemma}\label{lem:grassmann-decomp}
  Suppose \(k<n\). The open complement of the closed immersion
  \(\Gr[n-1]{k-1}\clim \Gr[n]{k}\) is the total space of a vector bundle of rank
  \(k\) over \(\Gr[n-1]{k}\).

  Dually, the open complement of \(\Gr[n-1]{k}\clim \Gr[n]{k}\) is the total
  space of a vector bundle of rank \(n-k\) over \(\Gr[n-1]{k-1}\).
\end{lemma}

\begin{lemma}
  Suppose \(k < n\) and let \(\transfer[n,k]\colon \one{} \to \Gr[n]{k}_+\)
  denote the Becker--Gottlieb transfer for \(\Gr[n]{k}\) in \(\SH^{\GL_{n-1}}\).
  Write \(i_{k-1}\colon \Gr[n-1]{k-1}\clim \Gr[n]{k}\)  and \(i_k\colon
  \Gr[n-1]{k}\clim \Gr[n]{k}\) for the inclusions\viktor{which inclusions
    exactly?}. Then
  \begin{align*}
    \transfer[n,k] &= i_{k-1,+}\circ\transfer[n-1,k-1] + (1 + \chi) i_{k,+}\circ\transfer[n,k-1] =\\
                   &= (1 + \chi')i_{k-1,+}\circ\transfer[n-1,k-1] + i_{k,+}\circ\transfer[n,k-1]
  \end{align*}
  \viktor{what's \(\chi\) and \(\chi'\)?}
\end{lemma}

\section{Proof of the Theorem}

The inclusions \(\GL_i\to\GL_n\) induce a filtration
\[
    \BGL_{0,+}\to[i_1] \BGL_{1,+} \to[i_2] \dots \to[i_n] \BGL_{n,+} \to
    \dots\to[i_m] \BGL_{m,+}.
\]
For \(k \leq n\) the block--diagonal inclusion \(i_{k,n}\colon\BGL_k\times\BGL_{n-k}\to \BGL_n\) has
homotopy fibers \(\GL_n/{\GL_k\times\GL_{n-k}}\). This homogeneous space is
\(\AA^1\)--homotopy equivalent to the Grassmannian \(\Gr[n]{k}\) of \(k\)--planes
in \(n\)--space which is smooth and proper. Consequently, \(i_{k,n}\) admits a
Becker--Gottlieb transfer \(\transfer[n,k]\colon\BGL_{n,+}\to \BGL_{k,+}\smashp
\BGL_{n-k,+}\). Write \(f_{n,k}\colon \BGL_{n,+}\to \BGL_{k,+}\) for the
composition
\[
  \BGL_{n,+} \to[{\transfer[n,k]}] \BGL_{k,+}\smashp\BGL_{n-k,+} \to[\proj] \BGL_{k,+}
\]
and \(\phi_{n,k}\) for the composition
\[
  \BGL_{n,+}\to[f_{n,k}] \BGL_{k,+}\to\BGL_k/{\BGL_{k-1}}.
\]

\begin{lemma}\label{lem:phi-comp}
  With notation as above, for \(k < n\) the compositions
\[
  \BGL_{n-1,+} \to[i_n] \BGL_{n,+} \to[f_{n,k}] \BGL_{k,+} \to \BGL_k/{\BGL_{k-1}}
\]
and
\[
  \BGL_{n-1,+} \to[f_{n-1,k}] \BGL_{k,+} \to \BGL_k/{\BGL_{k-1}}
\]
coincide.
\end{lemma}

\begin{proof}[{Proof of \autoref{thm:main}}]
  Proceeding by induction on \(n\), assume that
\[
  \Phi = \bigvee_{k=0}^{n-1} \phi_{n-1,k}\colon \BGL_{n-1,+} \to \bigvee_{k=0}^{n-1} \BGL_k/{\BGL_{k-1}}
\]
is a \(\PP^1\)--stable equivalence. Because of \autoref{lem:phi-comp} we have a
commutative diagram
\[
  \begin{tikzcd}
    {} & \BGL_{n,+}\ar[dr, "\Phi'"] & {} \\
    \BGL_{n-1,+} \ar[rr, "\Phi"'] \ar[ur, "i_n"] & {} & \displaystyle\bigvee_{k=0}^{n-1}\BGL_k/{\BGL_{k-1}}
  \end{tikzcd}
\]
where \(\Phi' = \bigvee_{k=0}^{n-1}\phi_{n,k}\). It follows that \(\Phi^{-1}\circ\Phi'\circ i_n \simeq \id\),
\ie~\(i_n\) admits a left inverse. This implies that we have a \(\PP^1\)--stable equivalence
\[
  \BGL_{n,+} \to[{(\Phi^{-1}\Phi') \vee \phi_{n,n}}] \BGL_{n-1,+} \vee \BGL_n/{\BGL_{n-1}}
\]
since \(\phi_{n,n}\) is by definition the canonical projection. Post-composing
with \(\Phi\vee{\id}\) then shows that the stable map \(\Phi'\vee\phi_{n,n}\colon
\BGL_{n,+}\to\bigvee_{k=0}^n \BGL_k/{\BGL_{k-1}}\) is a \(\PP^1\)--stable
equivalence as well.
\end{proof}

\printbibliography

\listoftodos
\end{document}
% Local Variables:
% TeX-master: t
% End:
